\documentclass[12pt]{article}

% --- Página y tipografía ---
\usepackage[letterpaper,margin=2.5cm]{geometry}
\usepackage[T1]{fontenc}
\usepackage[utf8]{inputenc} % pdfLaTeX
\usepackage{lmodern}
\usepackage{microtype}

% --- Imágenes ---
\usepackage{graphicx}

% --- Espaciado ---
\usepackage{setspace}
\setlength{\parindent}{0pt}
\setstretch{1.2}

\begin{document}
\thispagestyle{empty}

% ===== Encabezado =====
\begin{minipage}[c]{0.18\textwidth}
    \centering
    \includegraphics[width=0.95\linewidth]{img/logo_usac.jpeg}
\end{minipage}
\hfill
\begin{minipage}[c]{0.60\textwidth}
    \small
    Universidad de San Carlos de Guatemala\\
    Escuela de Ciencias Físicas y Matemáticas\\
    \textbf{José Manuel Noriega Pineda}\\
    Carnet: 202505312\\
    Programación 1
\end{minipage}
\hfill
\begin{minipage}[c]{0.18\textwidth}
    \centering
    \includegraphics[width=1.4\linewidth]{img/logo_ecfm.jpg}
\end{minipage}

\vspace{0.5cm}

\noindent\rule{\textwidth}{1.2pt}

\vspace{0.3cm}

% ===== Título =====
\begin{center}
    {\Large\scshape Mi rama de la Física}
\end{center}

\vspace{0.2cm}

\begin{center}
    \small\scshape 06 de febrero de 2026
\end{center}

\vspace{0.3cm}

\noindent\rule{\textwidth}{1.2pt}

\vspace{0.8cm}

% ===== Ensayo =====
\section*{Ensayo}

Mi área de interés dentro de la Física es la Física Nuclear, la cual se enfoca en el estudio de la estructura, propiedades e interacciones del núcleo atómico. Esta rama de la Física resulta fundamental para comprender procesos básicos de la materia y la energía, así como para el desarrollo de aplicaciones tecnológicas como la producción de energía nuclear, la medicina nuclear y el estudio de la radiación.

La investigación en Física Nuclear involucra sistemas altamente complejos, donde intervienen interacciones fundamentales y fenómenos que no siempre pueden resolverse de forma analítica. Por esta razón, la programación se convierte en una herramienta esencial, ya que permite modelar matemáticamente estos sistemas y simular procesos nucleares mediante métodos numéricos.

El uso de la programación facilita la resolución de ecuaciones diferenciales, el análisis de grandes volúmenes de datos experimentales y la visualización de resultados de manera clara y eficiente. Además, permite automatizar cálculos repetitivos y reducir errores, optimizando el tiempo y la precisión en el trabajo científico.

En conclusión, la programación es una habilidad indispensable dentro de la Física Nuclear, ya que complementa la formación teórica y experimental del físico, ampliando las posibilidades de investigación y contribuyendo al avance del conocimiento científico.

\end{document}
